\section{Parallel Methods}

In this section, we introduce two parallel methods for computing the interpolation. Both of them interpolate $f_z$ logarithmically. The only difference is the way that they compute $f_{\overline{z}}$. 

When we interpolate $f_z$ logarithmically, this means that we interpolate $\arg f_z$ linearly. 

\subsection{Logarithmic Interpolation of $f_z$}

In order to interpolate $f_z$ logarithmically, we follow this formula where the input are $f_z^0, f_z^1$. It is expressed as:

\begin{align*}
f_z^t &= (f_z^0)^{1-t} (f_z^1)^t \\
&= e^{(1-t) \log f_z^0 + t \log f_z^1} \\
&= |f_z^0|^{1-t} |f_z^1|^{t} e^{i \left( (1-t) \arg(f_z^0) + t \arg(f_z^1) \right)}
\end{align*}

Note that in the second equation, we need to precisely define what our $\log$ function is. 



\subsection{Bounding Conformal Distortion}
% not super necessary?
% basically just tries to mathematically prove some bounds
% but we are mostly concerned with application

The second complex dilation $\nu$ is defined as:

\begin{equation}
\nu = \frac{\overline{g_{\overline{z}}}}{g_z}
\end{equation}

where $g$ is a planar mapping. 


This means that we can calculate $f_{\overline{z}}$ as 

$$f_{\overline{z}} = \nu g_z$$

Now if we linearly interpolate $\nu$ with respect to time, we obtain:

$$\nu^t = (1-t) \nu^0 + t \nu^1$$
$$f_{\overline{z}}^t = \overline{\nu^t f_z^t}$$

Note that the derived formula for $f_{\overline{z}}^t$ is anti-holomorphic as expected.

This is the first method, also known as the $\nu$ method.

\subsection{Interpolating Stretch Direction}

Short summary: We want to interpolate the stretch direction. This property is our 8th property and is not fulfilled by the $\nu$ method.

\subsection{Introducing $\eta$}

For this reason, we introduce the second method, called the $\eta$ method. For a planar mapping $g$, we define 

$$ \eta = g_{\overline{z}} \overline{g_z} = \mu |g_z|^2 $$

We linearly interpolate $\eta$ in order to obtain the equations:

$$\eta^t = (1-t) \eta^0 + t \eta^1$$
$$f_{\overline{z}}^t = \frac{\eta^t}{ \overline{f_z^t} }$$

This method allows us to interpolate $f_{\overline{z}}^t$ with the 8th property. 




\subsection{Scaling $\eta$}

The unscaled $\eta$ variant might not have the following properties:

\begin{enumerate}
	\item local injectivity
	\item geometric distortion bounds % what does this mean?
\end{enumerate}

In this case, we may scale our linearly interpolated $\eta(t)$ according to the following formula:

$$\widetilde{\eta}(t) = \rho(t) \eta(t)$$

For a particular choice of $\rho(t)$, we can gain local injectivity as well as geometric distortion bounds. To give a little intuition, suppose our function $\rho(t)$ approaches $0$ for all $t$. Then $f_{\overline{z}}^t$ approaches $0$ as well, signifying that the mappings become more and more conformal, lowering conformal distortion bounds.

One particular issue with globally scaling $\eta(t)$ by $\rho(t)$ is that there might be a ``qualitative non-locality," to quote the text. Essentially, the scaling needed in one portion might cause another portion to be scaled to near conformality even when not desired.





